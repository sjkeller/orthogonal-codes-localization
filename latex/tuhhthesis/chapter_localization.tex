\chapter{Localization}

\section{Peak detection}

The received signal, consisting of summed delayed signals, cross-correlated by every anchor. If the signal is not reflected the peak in cross-correlation would be obvious. But because by the introduction of noise and water reflections a higher rate of similar peaks appear. To suppress these effects a CA-FAR Algorithm \cite{rohling11} is applied to only detect the first reflected peak resulting in lower false alarms of peaks. \\
CA-FAR works by using multiple values intervals. The most outer one could be described as a train bin and is used to get an estimation of the signals noise. Especially CA-FAR uses averaging to estimate the noise by measured cells. The bordering bin, defined as the guard cells, is used to reduce self-interference of the peaks. Thus, increasing window sizes $W$ results in better noise estimating but overall detectability is still limited by the sample rate  \cite{rohling11}\cite{radarbasics}. By knowledge of measured peak widths a optimal guard interval can be figured.\\
The calculated threshold is than scaled by a formula depending on the false alarm rate. The higher the false alarm rate, the weaker high amplitude peaks gets included by the estimated threshold \ref{eq:cff}.\\
%Because Cell Averaging shows not satisfactory results in sensitive multi targets examples, the noise estimation could be enhanced by a sorting average applied to the %train interval. That principle is defined as CO-FAR \cite{rohling11}. 
%\begin{center}
%	\begin{tabular}{|c|c|}
%			\hline
%			candidate sample & $i$\\
%			\hline
%			guard interval (half) & $\mathcal{G}$ \\
%			\hline
%			train interval (half) & $\mathcal{T}$ \\
%			\hline
%			false alarm rate & $\eta$\\
%			\hline
%		\end{tabular}
%	\linebreak
%\end{center}
%
%\begin{equation}
%	Threshold(i)_x=
%	\cfrac{\alpha}{2\mathcal{T}}
%	\left[ \sum_{j=i-\left( \mathcal{G}+\mathcal{T}\right) }^{i+\left( \mathcal{G}+\mathcal{T}\right)}x(j) - \sum_{j=i-\mathcal{G}}^{i+\mathcal{G}}x(j) \rigbest NAht] 
%	\label{eq:cft}
%\end{equation}
%\begin{equation}
%	\text{scale factor}~~~
%	\alpha=2\mathcal{T}\left( \eta^{-1/{2\mathcal{T}}}-1\right) 
%	\label{eq:cff}
%\end{equation}
\begin{figure}[h]
	\includegraphics[width=\linewidth]{images/peakdet}
	
	\caption{CFAR threshold peak detection procedure \cite{rohling11}}
	\label{fig:simsig}
\end{figure}

\begin{equation}
	T=S\cdot Z,~~~Z_{CA}=\sum_{i=1}^{W}\dfrac{1}{W}x_{train}
	\label{eq:cff}
\end{equation}


\section{Explicit Position calculation}

The initial condition for the localization are four anchors $S_i$ with thir coordinates $\{x_i,y_i,z_i\}$ and the target $S$ which is to be located. By multiplying relative delays by the speed of sound $c$ which is approximately set to $1500\,\frac{m}{s}$, the distance $d_{ij}$ between the reference anchor $S_0$ and $S_i$ is calculated.\\
\begin{equation}
	d_{ij}=c\cdot\tau_{ij}=c\cdot (t_i-t_j),~~~\text{absolute delays } t_k,~k\in \{0,1,2,3\}
\end{equation}
By these values a Matrix $A$ and vector $\vec{b}$ is created. Thus, a System $A\cdot \vec{x}=\vec{b}$ is established. This linear system can be solved by using the inverse method if Matrix $A$ has full rank. Otherwise least squared can be used but yields undesirable results. In three dimensional Space five anchors are needed for full rank. The solution $\vec{x}$ are the coordinates $\{x,y,z\}$ of the target $S$ \cite{yang11}.

\begin{equation}
	A=\left[
	\begin{array}{cccc}
		x_0-x_1 & y_0-y_1 & z_0-z_1 & d_{01}\\
		x_0-x_2 & y_0-y_2 & z_0-z_2 & d_{02}\\
		x_0-x_3 & y_0-y_3 & z_0-z_3 & d_{03}\\
		x_0-x_4 & y_0-y_4 & z_0-z_4 & d_{04}\\
	\end{array}
	\right]
\end{equation}
\begin{equation}
	\vec{b}=\cfrac{1}{2}\left[
	\begin{array}{c}
		x_0^2-x_1^2+y_0^2-y_1^2+z_0^2-z_1^2+d_{01}^2\\
		x_0^2-x_2^2+y_0^2-y_2^2+z_0^2-z_2^2+d_{02}^2\\
		x_0^2-x_3^2+y_0^2-y_3^2+z_0^2-z_3^2+d_{03}^2\\
		x_0^2-x_4^2+y_0^2-y_4^2+z_0^2-z_4^2+d_{04}^2\\
	\end{array}
	\right]
	,~~~
	\vec{x}=\left[
	\begin{array}{c}
		x\\
		y\\
		z\\
		||S-S_0||_2\\
	\end{array}
	\right]
\end{equation}
Theoretically an system $A$ of lower rank can be solve by least squares. Having said this, positions calculated by that approach could not satisfy the demands of 3D localization.

\begin{equation}
	\vec{x}=(A^T A)^{-1} A^T\vec{b}
\end{equation}

\section{GPS like localization method}

Because of restrictions introduced in the the first localization method an alternative localization procedure is probed. Thus a position calculation by four anchors is alternatively pursued.\\
\begin{equation}
	x_{ji}:=x_j-x_i,~~~~~
	y_{ji}:=y_j-y_i,~~~~~
	z_{ji}:=z_j-z_i~~~~~
\end{equation}
By rearranging the derivation of hyperbola intersections the following substitutes can be defined. 
\begin{equation}
	\mathbf{A}=\cfrac{d_{02}x_{10}-d_{01}x_{20}}{d_{01}y_{20}-d_{02}y_{10}},~~~
	\mathbf{B}=\cfrac{d_{02}z_{10}-d_{01}z_{20}}{d_{01}y_{20}-d_{02}y_{10}}
\end{equation}	
\begin{equation}
	\mathbf{C}=\cfrac{d_{02}\left(d_{01}^2+x_0^2-x_1^2+y_0^2-y_1^2+z_0^2-z_1^2\right)-d_{01}\left(d_{02}^2+x_0^2-x_2^2+y_0^2-y_2^2+z_0^2-z_2^2\right)}{2\left(d_{01}y_{20}-d_{02}y_{10}\right)}
\end{equation}	

\begin{equation}
	\mathbf{D}=\cfrac{d_{23}x_{12}-d_{21}x_{32}}{d_{21}y_{32}-d_{23}y_{12}},~~~
	\mathbf{E}=\cfrac{d_{23}z_{12}-d_{21}z_{32}}{d_{21}y_{32}-d_{23}y_{12}}
\end{equation}	
\begin{equation}
	\mathbf{F}=\cfrac{d_{23}\left(d_{21}^2+x_1^2-x_1^2+y_1^2-y_1^2+z_0^2-z_1^2\right)-d_{21}\left(d_{23}^2+x_1^2-x_2^2+y_1^2-y_2^2+z_1^2-z_2^2\right)}{2\left(d_{21}y_{32}-d_{23}y_{12}\right)}
\end{equation}	

\begin{equation}
	\mathbf{G}=\cfrac{\mathbf{E}-\mathbf{B}}{\mathbf{A}-\mathbf{D}},~~~
	\mathbf{H}=\cfrac{\mathbf{F}-\mathbf{C}}{\mathbf{A}-\mathbf{D}},~~~
	\mathbf{I}=\mathbf{A}\cdot \mathbf{G}+\mathbf{B},~~~
	\mathbf{J}=\mathbf{A}\cdot \mathbf{H}+\mathbf{C}
\end{equation}	

\begin{equation}
	\mathbf{K}=d_{02}^2+x_0^2-x_2^2+y_0^2-y_2^2+z_0^2-z_2^2+2x_{20}\mathbf{H}+2y_{20}\mathbf{J}
\end{equation}	

\begin{equation}
	\mathbf{L}=2\left(x_{20}\mathbf{G}+y_{20}\mathbf{I}+z_{20}\right)
\end{equation}	

\begin{equation}
	\mathbf{M}=4d_{02}^2\left(\mathbf{G}^2+\mathbf{I}^2+1\right)-\mathbf{L}^2
\end{equation}	

\begin{equation}
	\mathbf{N}=8d_{02}^2\left[\mathbf{G}\left(x_0-\mathbf{H}\right)+\mathbf{I}\left(y_0-\mathbf{J}\right)+z_0\right]+2\mathbf{L}\cdot\mathbf{K}
\end{equation}	

\begin{equation}
	\mathbf{O}=4d_{02}^2\left[\left(x_0-\mathbf{H}\right)^2+\left(y_0-\mathbf{J}\right)^2+z_0^2\right]-\mathbf{K}^2
\end{equation}	

A downside of this approach is the uncertainty of position $z$. Thus, additional information on bounds is necessary. The target won't get above sea level. Consequently, at least one boundary $z_{surface}$ which acts like a maximum can be set. The minimum value $z_{ground}$ can be assumed as the lowest position achievable underwater. 
\begin{equation}
	z_{a,b}=\cfrac{\mathbf{N}}{2\mathbf{M}}\pm\sqrt{\left(\cfrac{\mathbf{N}}{2\mathbf{M}}\right)^2-\cfrac{\mathbf{O}}{\mathbf{M}}}
\end{equation}	

\begin{equation}
	z=\min\left\{\max\left\{z_a,~z_b,~z_{surface}\right\},~z_{ground}\right\}
\end{equation}	

The resulting x and y values of our target can then be calculated by the following formula using the selected $z$.

\begin{equation}
			\vec{x}
	=\left[
	\begin{array}{c}
		x\\
		y\\
		z
	\end{array}
	\right]
=\left[
	\begin{array}{c}
		\mathbf{G}z+\mathbf{J}\\
		\mathbf{I}z+\mathbf{H}\\
		z
	\end{array}
	\right]
\end{equation}	
