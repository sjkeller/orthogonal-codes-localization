%%
%% DOCUMENT TYPE
%%

% general options:
% - inputenc        file encoding (should be "utf8" in most cases)
% - de/en           language of your work (influence pre-defined tokens)
% - declaration     adds the mandatory statutory declaration for theses
% - abstract        adds the abstract (from file "prelude_abstract.tex")
% - acknowledgment  adds an acknowledgment (from file "prelude_acknowledgment.tex")
%                   it is a nice gesture to personally thank people who
%                   supported you during your work.
% - symbollist      adds a list of symbols (from file "prelude_symbols.tex")
% - figurelist      adds and automatically creates a list of figures 
% - tablelist       adds and automatically creates a list of tables
% - index           generates an index based on the package "makeidx", please
%                   refer to its documentation for usage on index markup
% - bibbacklinks    adds backlinks from bibliography to the pages, where the
%                   corresponding entry is used (cited)
% - gray            make a gray-style version of the thesis report
%
% PhD thesis specific options
% - cv              adds your cv
% - publishsize     changes the page size from A4 to A5 for print publishing
%                   (please change the font size to 9pt, if you use this option)
% - approved        use this option, after your thesis has been formally approved
%                   (this will change the front page to meet formal/legal requirements)
% - ownpub          adds a second bibliography (from file "ownpub.bib") for your own
%                   publications related to the PhD thesis. According to the latest
%                   examination regulations, own work should be part of the regular
%                   bibliography (this option is hence obsolete)

\documentclass[en,abstract,acknowledgment,symbollist,inputenc=utf8]{tuhhthesis}

\usepackage{tuhhlistings}
\usepackage{multirow}
%%
%% SETUP BLOCK
%%

% thesis type, must be one of the following
% - projectwork
% - bachelorthesis
% - masterthesis
% - diplomathesis
% - phdthesis
\setthesistype{projectwork}

% your full name as printed on any official document (e.g., passport)
\author{Sergej Keller}

% the official title of your work (*must* match the filed title)
\title{Orthogonal Codes for Acoustic UnderwaterLocalization}

% the institution of the first examiner (refer to tuhhlangnames.def)
\institute{InstAutonomousCPS}

% date of submission as DD.MM.YYYY
\submitdate{01.04.2022}

% your matriculation number (for anything but PhD thesis)
\matrnumber{50484}

% PhD thesis only
%\setSexOfAuthor{male}
%\setBirthplace{Hamburg, Deutschland}
%\setPhDType{ing}

% your course of studies
\course{Informatik-Ingenieurwesen}

% full name and affiliation of first and second examiner
\examinerFirst{Prof. Dr.-Ing. Bernd-Christian Renner}{Institute of Autonomous Cyber-Physical Systems\newline Hamburg University of Technology}
\examinerSecond{Prof. Dr. Volker Turau}{Institute of Telematics\newline Hamburg University of Technology}

\supervisorFirst{Christoph Weyer}{Institute of Telematics, Hamburg University of Technology}
%\supervisorSecond{Volker Turau}{Institute of Telematics, Hamburg University of Technology}

% optional: print the TUB document number on title page
% this only applies, if the document is formally publish under
% a TUB document number
%\tubdoknumber{4711}


% Curriculum Vitae
% only needed for thesis type PhD
%\usepackage[]{currvita}
%\setlength{\cvlabelwidth}{50mm}
%\renewcommand*{\cvlistheadingfont}{\normalfont\sffamily\large\color{tuhh_blue}}
%\renewcommand*{\cvlabelfont}{\normalfont\rmfamily\normalsize\color{tuhh_darkgray}}



%%
%% CONTENT AREA
%%

% mathematical symbols
\input{setup_math}


\begin{document}

\chapter{Introduction}
\section{Motivation}
\section{Setup}
\section{Principle}
\chapter{PN and orthogonal sequences}

To attain a higher level of localization accuracy, there are two primary goals that must be pursued.\\
First, the code used for underwater localization should have an auto-correlation function that approaches a Dirac impulse. This is important because it allows for more efficient detection through the use of correlation techniques.\\
The second factor to consider is the cross-correlation properties of the code. It is essential that these attributes meet certain criteria in order to improve separation from other sequences. Mathematically speaking, this means that the codes should be orthogonal to each other, or at least approaching orthogonality. This will be particularly useful in real-world scenarios where noise, reflections, and other artifacts may be present.\\
In summary, by striving to achieve both of these objectives, it is possible to significantly improve the localization accuracy.

\section{Pseudo-random codes}

There are a couple of techniques to generate PN sequences. Most of these methods use linear feedback shift registers to generate the codes by an initial condition or seed value. In this project I will concertize my research on gold codes, kasami codes and the basic m-sequences which are used for generating gold codes. These types are all based on linear shift registers.\\
M-sequences are defined as binary PN codes, which are generated by linear shift registers with feedback. The sequences are periodic, and contain an equal number of zeros and ones \cite{proakis08}. 
Maximum length sequences need to fulfill certain criteria.  First its length is defined by $N=2^n - 1$ where $n$ is the maximum degree of the generator polynomial $f(X)$ \cite{sarwate80}.
 \begin{equation}
	 \lvert u\rvert=2^n-1=N,~~~\text{from polynomial}~~h(x) \text{of degree}~~n
\end{equation}
\begin{equation}
	\dfrac{N}{gcd(N,q)=N},~~~\text{from decimation polynomials}~~\widetilde{h(x)}
\end{equation}
Second the cross-correlation between m-sequences must take three values only, which are $-1$, $-t(n)$, $t(n) - 2$. With it $t(n)$ is defined by $1+2^{\lfloor0.5(n+2)\rfloor}$ \cite{sarwate80}.
If every pair of m-sequences is a preferred pair, they form a maximal connected set and these sets have a limited carnality. Experiments from Gold and Koptizke showed that the number of such connected pairs is limited. Between degrees $$$$ \cite{gold65}. To get an m-sequence we need a primitive polynomial. 

%\fignoframe{images/lfsr}{Basic structure of an LFSR (Linear Feedback Register). \cite{proakis08}}{fig:framelessFigures}

\subsection{Gold Codes}

Because of not optimal cross-correlation properties m-sequences alone are not applicable for the project. But if these type of codes are combined their correlation qualities can change. Gold Codes are m-sequences where two of them with same length are modulo-2 summed. \cite{proakis08} \\
Recent research shows that some gold codes have high similarity to a Gaussian random variable \cite{merrifield}. 

\begin{equation}
Gold(u,v)=\{u,v,u\oplus v,u\oplus(v \ll1),\dots,u\oplus(v\ll N-1)\}
\end{equation}

%\fignoframe{images/gold}{LFSR structure of preferred generator polynomial of degree 13. \cite{merrifield}}{fig:framelessFigures}

\subsection{Kasami Codes}

Kasami sequences are constructed in the similar fashion by using m-sequences with the exception that a second sequence, which is used in the modulo sum, is formed by decimating the default m-sequence by  $2^{m/2}$ \cite{proakis08} \cite{sarwate80} \cite{peterson72}. Thus, only one generator polynomial is required.

\begin{equation}
w=u[2^{N/2}+1]=\{u_1,\dots, u_i, \dots,u_{N}|\text{take every }i\text{-th bit of u}\} 
\end{equation}
\begin{equation}
Kasami(u)=\{u,u\oplus w,u\oplus(w \ll1),\dots,u\oplus(w\ll2^{N/2}-2)\}
\end{equation}


\section{Comparison}


For the localization process by orthogonal codes certain criteria needs to be met, which were named in the first chapter. To compare the before explained code types three measures are introduced. \\ 
The first one is the peak to side-lobe ratio (PSR) \ref{eq:psr}. This measure is defined by subtracting the mean from the peak of the auto-correlation. Then this value get divided by the standard deviation of the same auto-correlation. A higher PSR value signifies a lower error between the auto correlation and the perfect Dirac resulting in better detection capability. The second one is the ratio between the auto-correlation peak and the maximum of the cross-correlation (ACR) \ref{eq:acr}. There a higher value indicates good code separation qualities.\\
The comparison is done by sampling pairs at degrees six to ten $N$ times from the set of random sequences. These pairs are then used for generating the wanted pseudo-random codes like gold or kasami. Afterwards for all three code types the shown measures are applied.  

%The last one is the correlation coefficient showing if there are similar anchros \ref{eq:coeff}, which would be a bad indicator.

\begin{equation}
PSR=\dfrac{max\{x_{ac}\}-\overline{x_{ac}}}{\sigma_{ac}}
\label{eq:psr}
\end{equation}

\begin{equation}
ACR=\dfrac{max\{x_{ac}\}}{max\{{x_{cc}\}}}
\label{eq:acr}
\end{equation}

%\begin{equation}
%\rho(a1,a2)=\dfrac{cov\{a1,a2\}}{\sigma_{a1}\cdot\sigma_{a2}}
%\label{eq:coeff}
%\end{equation}
%\begin{figure}[h]
%	\includegraphics[width=8cm]{images/matlabplots/mseq}
%
%	\caption{Maximum Length Sequence evaluation}
%\end{figure}
%
%\begin{figure}[h]
%	\includegraphics[width=8cm]{images/matlabplots/gold}
%
%	\caption{Gold sequence evaluation}
%\end{figure}
%
%\begin{figure}[h]
%	\includegraphics[width=8cm]{images/matlabplots/kasami}
%
%	\caption{Kasami sequence evaluation}
%\end{figure}
%%\fignoframe{images/matlabplots/mseq}{Basic structure of an LFSR (Linear Feedback Register). \cite{proakis08}}{fig:framelessFigures}
%%\fignoframe{images/matlabplots/gold\_10ms}{Basic structure of an LFSR (Linear Feedback Register). \cite{proakis08}}{fig:framelessFigures}
%%\fignoframe{images/matlabplots/kasami\_10ms}{Basic structure of an LFSR (Linear Feedback Register). \cite{proakis08}}{fig:framelessFigures}
%\section{Results}
In this evaluation of data, three types of codes were compared. Maximum length sequences, Gold Codes, and Kasami Codes. The performance of each code was assessed using two ratios, the ACR and the PSR.\\
From preferred polynomial all possible maximum length sequences, gold sequences and kasami sequences are generated. Then both measures are applied on the cross-correlation and auto-correlation functions of the random codes. The PSR and ACR measures are plotted against the used polynomials. Also the best case of PSR and ACR are plotted by their given correlation function.\\
Maximum length sequences hold the best auto-correlation properties in comparison to its competitors. But it shows peaks in its cross-correlation, making it a rather bad option for orthogonal separation. The kasami sequence has a way better cross-correlation but still a small peak. The clear winner are gold codes because of the good auto-correlation and very good cross-correlation properties \ref{fig:eva}. Its auto-correlations lags a bit behind its competitors but orthogonality is as much as important. 
%
%\begin{figure}[h]
%	\centering\includegraphics[width=13cm]{images/mseqevaacr}
%	
%	\caption{Evaluation of m-sequences by AC ratio}
%	\label{fig:eva}
%\end{figure}
%
%\begin{figure}[h]
%	\centering\includegraphics[width=13cm]{images/mseqevapsr}
%	
%	\caption{Evaluation of m-sequences by PS ratio}
%	\label{fig:eva}
%\end{figure}

\begin{figure}[h]
	\centering\includegraphics[width=14cm]{images/goldevaacr}
	
	\caption{Evaluation of gold sequences by AC ratio}
	\label{fig:eva}
\end{figure}

\begin{figure}[h]
	\centering\includegraphics[width=14cm]{images/goldevapsr}
	\caption{Evaluation of gold sequences by PS ratio}
	\label{fig:eva}
\end{figure}
To get more valuable data, elements from each set of codes are sampled uniformly ($N=1000$) and afterwards the evaluation parameters are applied.
The results yields that Gold Codes had the least increase in PSR, but were the second best at ACR. Kasami Codes were only slightly better in both PSR and ACR than Gold Codes, but had a smaller set of codes available. Maximum length sequences had the highest PSR, but the worst ACR.\\
Based on these findings, it can be concluded that Gold Codes are the best choice for this application. While maximum length sequences had the highest PSR, they performed poorly in terms of ACR. Gold Codes, on the other hand, had a good balance of performance in both ratios, and also had a large set of codes available. In addition, Gold Codes demonstrated better cross-correlated detection compared to maximum length sequences.
\begin{figure}[h]
	\centering\includegraphics[width=14cm]{images/degCompEva}
	
	\caption{Evaluation by relative PSR for degrees 6 to 11}
	\label{fig:eva}
\end{figure}

%\begin{figure}[h]
%	\centering\includegraphics[width=13cm]{images/degAcrEva}
%	
%	\caption{Evaluation by relative ACR for degrees 6 to 11}
%	\label{fig:eva}
%\end{figure}


\chapter{Signal Processing}
The signal processing is separated into two sections. First the code needs to be transformed into the base-band. After that a spectrum shift to a specific transmission band is realized. Afterwards the exported signal is put through an simulator which adds reflection noise. The result is than again imported and gets reverse spectrum shifted. At the end a peak detection method is used to ignore reflection peaks.
\section{Base-Band}
Before a spectrum shift is applied to the signal, the bandwidth needs to be bounded. Otherwise absolute code bits would result in theoretically infinite frequencies which are impossible to implement for transmission. A raised cosine filter is therefore applied to remove all unwanted frequencies above a certain threshold. The base-band for our application is $20kHz$. Thus, our symbol length is set to $1/{20kHz}$. A appropriate roll-off coefficient of $0.125$ is picked. The resolution of cosine needs to be high enough to include at least a couple of periods.  A whole cosine is not tangible because its periodic and therefore infinite in time.
\section{Transmission Band}
Now the spectrum is ready to be shifted by the given transmission frequency $f_c$. The resulting signal could hold imaginary parts, hence we only move the real part further in processing. 
\begin{equation}
 	x_{TB}(t)=Re\{x_{BB}(t)\cdot e^{-2\pi j f_c t}\}
\end{equation}
\section{Simulation}
The simulation consists of a watermark benchmark \cite{watermark15} and a additive GWN generated by a desired SNR between $-20dB$ and $20dB$. From the general equation of the Signal to Noise Ratio we derive our noise standard deviation by transforming this ratio.
\begin{equation}
	SNR=\cfrac{P_{Signal}}{P_{Noise}}=\cfrac{\mathbb{E}(S^2)}{\mathbb{E}(N^2)}=\cfrac{\sum S^2}{N\cdot\sigma_N^2}~~\Leftrightarrow~~\sigma_N=\sqrt{\cfrac{N}{SNR\cdot\sum S^2}}
\end{equation}
\section{Peak detection}
The received signal, consisting of summed delayed signals, cross-correlated by every anchor. If the signal was not reflected the peak in cross-correlation would be obvious. But because of the introduction by noise and water reflections more peaks could appear. To suppress these effects a CA-FAR Algorithm \cite{rohling11} is applied to only detect the first reflected peak resulting in true results. 
\begin{center}
\begin{tabular}{|c|c|}
	\hline
	candidate sample & $i$\\
	\hline
	guard interval (half) & $\mathcal{G}$ \\
	\hline
	train interval (half) & $\mathcal{T}$ \\
	\hline
	false alarm rate & $\eta$\\
	\hline
\end{tabular}
\linebreak
\end{center}

\begin{equation}
	\text{Threshold}(i)_x=
	\cfrac{\alpha}{2\mathcal{T}}
	\left[ \sum_{j=i-\left( \mathcal{G}+\mathcal{T}\right) }^{i+\left( \mathcal{G}+\mathcal{T}\right)}x(j) - \sum_{j=i-\mathcal{G}}^{i+\mathcal{G}}x(j) \right] 
\end{equation}
\begin{equation}
	\alpha=2\mathcal{T}\left( \eta^{-1/{2\mathcal{T}}}-1\right) 
\end{equation}
% The Chapters
%\input{chapter_Introduction}
%\input{chapter_Structure}
%\input{chapter_TheClass}
%\input{chapter_Figures}
%\input{chapter_Style}


% Bibliography
% if you have cited papers that are not referenced, but important for your work,
% uncommented the following line; however, this should generally by unnecessary
% and hints at improper citing.
%\nocite{*}
\tuhhbibliography{thesis}


% Appendix
% Feel free to add additional appendix chapters (e.g., measurement setups, etc.)
\begin{tuhhappendix}
  \input{appendix_CD-Content}
\end{tuhhappendix}


% The End
\end{document}
