\chapter{PN and orthogonal sequences}

There are two main goals need to be pursued for receiving higher localization accuracy.\\    
First the code which is used for the under water localization needs its auto-correlation approaching a Dirac impulse. Thus one get advantageous detection capabilities.\\
The next factor are cross-correlation properties, which should meet certain criteria for improving the separation from other sequences. These will come in handy if noise, reflections and other artifacts emerge.

\section{Pseudo-random codes}

There are a couple of principles to generate PN sequences. Most of these methods use linear feedback shift registers to generate the codes by an initial condition. In this project I will concertize my research on gold codes, Kasami-Codes and the classical m-sequences which are also used for generating gold codes.\\
M-sequences are binary PN codes, which are generated by linear shift registers with feedback. The sequences are periodic and contain the same number of zeros and ones \cite{proakis08}. 
M-sequences need to fulfill certain criteria.  First its length is defined by $N=2^n - 1$ where $n$ is the maximum degree of the generator polynomial $f(X)$ \cite{sarwate80}.
Second the cross-correlation between m-sequences needs to take three values only, which are $-1$, $-t(n)$, $t(n) - 2$. With it $t(n)$ is defined by $1+2^{\lfloor0.5(n+2)\rfloor}$ \cite{sarwate80}.
If every pair of m-sequences is a preferred pair, they form a maximal connected set and these sets have a limited cardinality. Experiments from Gold and Koptizke showed that the number of such connected pairs is limited. Between degrees $$$$ \cite{gold65}.

\fignoframe{images/lfsr}{Basic structure of an LFSR (Linear Feedback Register). \cite{proakis08}}{fig:framelessFigures}

\subsection{Gold Codes}

Because of not optimal cross-correlation properties m-sequences alone are not applicable for the project. But if these type of codes are combined their correlation qualities can change. Gold Codes are m-sequences where two of them with same length are modulo-2 summed. \cite{proakis08} \\
The gold code which has the highest similarity to Gaussian random variable has a degree of 13. Thus, a generator polynomial pair of $x^13+x^4+x^3+x+1$ and $x^13+x^12+x^10+x^9+x^7+x^6+x^5+x+1$ is chosen \cite{merrifield} .

\fignoframe{images/gold}{LFSR structure of preferred generator polynom of degree 13). \cite{merrifield}}{fig:framelessFigures}

\subsection{Kasami Codes}

Kasami sequences are constructed in the same fashion by using m-sequencs. But now the second sequence which is used in the modulo sum is formed by decimating the default m-sequence by  $2^{m/2}$ \cite{proakis08} \cite{sarwate80} \cite{peterson72}. 

\subsection{Comparison}

\fignoframe[width=\paperwidth]{images/matlabplots/mseq\_10ms}{Basic structure of an LFSR (Linear Feedback Register). \cite{proakis08}}{fig:framelessFigures}
\fignoframe{images/matlabplots/gold\_10ms}{Basic structure of an LFSR (Linear Feedback Register). \cite{proakis08}}{fig:framelessFigures}
\fignoframe{images/matlabplots/kasami\_10ms}{Basic structure of an LFSR (Linear Feedback Register). \cite{proakis08}}{fig:framelessFigures}

